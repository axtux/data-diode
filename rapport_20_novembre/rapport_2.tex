\documentclass[a4paper,11pt]{article}
\usepackage[utf8]{inputenc}
\usepackage{textcomp}
\usepackage{lmodern}
\usepackage{listings}
\usepackage{graphicx}
\usepackage{listings}
\usepackage{color}
\definecolor{lightgray}{rgb}{0.9,0.9,0.9}
\definecolor{darkgray}{rgb}{0.4,0.4,0.4}
\definecolor{purple}{rgb}{0.65, 0.12, 0.82}
\usepackage{url}
\usepackage[top=3cm,bottom=3cm,left=3cm,right=3cm]{geometry}

\title{Royal Military Academy\\
	INFO-Y113 --- Management of Security: \\
	Concept Of Operations v2 \& Risk Analysis}

\author{DANHIER Pierre, LECOCQ Alexis, NYAKI Loïc}

\begin{document}
\maketitle
\newpage
\tableofcontents

\newpage

\section{Introduction}
In recent years, cyber-security has become a primary concern for companies all over the world. No matter the size of the company, data often represent the heart of their business and whether the concern is the secrecy of intellectual property, or users' privacy, the theft of private data bears a huge cost for companies. Be it a monetary cost (lawsuits, fines) or a reputation cost (loss of trust, public outrage). In the case of government agencies, states secrets and other classified information could be stolen by a foreign nation, possibly leading to the loss of lives in conflict zones, loss of political leverage on the international scene, domestic political turmoil and scandals or simply public embarrassment.\\

When trying to protect these sensitive data, a common measure is be to physically isolate the network from the internet, by creating an \textit{air gap}. Acting this way ensures that the data from the network is inaccessible from the outside world. The main issue with this method is that inevitably, some external data or files will at some point need to be imported into the secure network, be it for software update, or simply because some files from the outside are necessary for the people working in the secure network. In that case, a manual import (via USB drive, by connecting an external laptop into the secure network, or by using some other data transfer device) will be necessary.\\

The problem with that method is that it can compromise the security of the secure network. For instance, the data that is manually transferred into the network may have been infected by a malware, or the secure network might already be infected by a virus. In both cases, there is a possibility for some malicious code to exfiltrate data, or to spread a virus outside, by secretly writing on the device that was originally used to import the data into the network, such as USB drives or laptops.\\

As a consequence, we need to build a solution that prevents data leaks while allowing the transfer of files from the outside network into the secure network.

\section{Goals of the project}
\label{sec:goals}
Goals define the general direction of what the organization aims to accomplish, in the long term. Here, we wish to design a system that accomplishes three main goals :

\begin{enumerate}
\item{Create a device that completely prevents the exfiltration of data from a secure network, while allowing data to be transferred from the outside world into that secure network.}
\item{Ensure the availability of the system. The down times should be as minimum as possible.}
\item{Allow specific users to manage and monitor this system through an administration web page.}
\end{enumerate}

In this project, the general solution is imposed and should be a data diode, which will be described in section~\ref{sec:data-diode}.

\section{Scope}
It is important to precisely identify the scope of this project, in relation to our goals and objectives.\\

Our solution is destined to be integrated in an existing system. As such, when considering the security of the system as a whole, we must identify which security aspects fall under our responsibility and which don't. 

\subsection{In scope}
The following elements are in scope, which means that it is our responsibility to make sure that the security of these elements is ensured.

\begin{itemize}
\item{The availability of the service}
\item{The confidentiality of user data and credentials when interacting with the data diode (see section \ref{sec:data-diode})}
\item{The impossibility to exfiltrate data from the secure network}
\item{The integrity of the data pushed through the data diode}
\end{itemize}

\subsection{Out of scope}
\label{sec:outscope}
The following elements are out of scope. This means that the security of these elements does not fall under our responsibility, but rather under the responsibility of the client, or another third party.

\begin{itemize}
\item{The physical access to the hardware, such as the power button or Ethernet cables}
\item{The physical integrity of the hardware}
\item{The electrical power source}
\item{The security within the secure network, such as the presence of malwares or other viruses}
\item{The behaviour of the employees allowed to interact with the diode}
\end{itemize}


\section{Data Diode}
\label{sec:data-diode}
To ensure the integrity and the confidentiality of the data within our system and the availability of the service, we are going to implement a \textit{data diode}.\\

Just like an electrical diode only conduct electrical current in one direction, a data diode is a networking device that only allows data to flow in one direction. It is composed of two physical servers: one server communicates with the outside network and the other one communicate with the secure network. The two servers are to one another by a single unidirectional fiber optics cable.\\

A fiber optic connection normally uses two cables: one for each direction. In the case of 	 data diode, only one cable is used, allowing the data to flow in one direction. The cable going in the other direction is physically cut. As a consequence, data going through a data diode can only flow in one direction, which is one of the goals required in section \ref{sec:goals}.\\

The one-way communication channel between the two sides of the data-diode forbids the use of usual TCP based protocols (such as HTTP or FTP), as TCP requires bi-directional communication between two parties. As data between the two sides of the data-diode can only flow in one direction, we need data to be send over a protocol that doesn't require bi-directional communication. This can be done by using UDP for the communication between our two servers.\\

However UDP comes with its own limitations, as it doesn't ensure the order at which the paquets arrive, nor does it manage packets loss. These issues will therefore need to be taken into account at the software level.\\


\begin{figure}
	\includegraphics[scale=0.7]{img/system.png}
	\caption{High level architecture of our data diode.}
\end{figure}


\subsection{Applications}
\subsubsection{JSON API}
We propose a JSON API to import files comming from connected devices to a secure network. To push data, a device needs to send a HTTP request to the outside machine with the basic authentication of the protocol in the headers and the JSON file in the body. When the request is received, the outside machine parse the HTTP request into a JSON file and send it with UDP to the secure side server. This machine will then store the file.

\subsubsection{Administration and User Management}
\label{sec:administration}
One of the goals of this project is the creation of a web interface to manage the data-diode. We will implement a web interface on the unsecure side that allows administrators to add new users or new administrators and a second web interface on the secure side that monitors the arrival of the UDP packets and creates an access to the data.

\subsubsection{Mockups}
\textbf{Unsecure side webpage}
\begin{center}
\begin{tabular}{cc}
\includegraphics[scale=0.5]{img/outsideLogin.png} & \includegraphics[scale=0.45]{img/outsideManagement.png}\\
\end{tabular}
\end{center} 

\textbf{Secure side webpage}
\begin{center}
\includegraphics[scale=0.5]{img/linkudp.png}
\end{center}



\subsection{Physical Architecture}
The data-diode is composed of two separate servers connected to each other through a unidirectional fiber optics cable. The server facing the outside network is called \textit{the sender}, the server facing the secure network is called \textit{the receiver}.\\

Each server contains the following components :
\begin{itemize}
\item{Two network interfaces: one for connecting to a network, and one for connecting to the other server}
\item{A fiber optic adapter, to translate the signal coming from the fiber optic cable from light into a signal that can be transmitted through an Ethernet port.}
\end{itemize}

\subsection{Software architecture}
The data-diode is composed of two servers, \textit{the sender} and \textit{the receiver}, which have different roles. 

\subsubsection{The Sender}
The \textit{sender} must be able to receive HTTP requests from the outside network and to transmit JSON files to the \textit{receiver} via UDP.\\

The sender must therefore be comprised of :
\begin{itemize}
\item{A UDP client}
\item{A web server}
\item{A php server}
\end{itemize} 

\subsubsection{The Receiver}
The \textit{receiver} must be able to receive UDP paquets and reconstruct them as a JSON file that will be placed in a database. \\

In summary, the receiver will contain :
\begin{itemize}
\item{A UDP server}
\item{A web server}
\item{A php server}
\end{itemize}

\subsection{Simulating a data-diode}
At first, instead of using a real data-diode, we are going to build the prototype of a data-diode by simulating the actual system through software. The unidirectional nature of the data transfer will be simulated by modifying the IP table in a way that will allow traffic in one direction, and drop all traffic going in the other direction.


\section{Users}
We identify two types of users : the administrators and the simple users.

\subsection{Simple Users}
Simple users are the data pushers using their connected devices.
\subsection{Administrators}
The role of the administrators is to manage the users and other administrators .
\section{Administration and Management}
\subsubsection{Administration of Simple Users}
Administrators can create and delete user accounts. If a user lost his password, he needs to contact an administrator who will give him a new one.

\subsubsection{Administration of other Administrators}
Administrators are able to create an administrator account, or to suspend the account of another administrator.
\subsection{data-diode Installation and Configuration}
\subsubsection{Installation}
The data-diode is provided to the client with the programs already installed, and a default administrator account. At first use, the password should be modified.
\subsubsection{Configuration}
The data-diode settings will be specified in a configuration file that can be modified manually.
\section{Risk analysis}
\subsection{Threat sources}
We identified several threat sources:\\
\begin{itemize}
\item Malicious user
\item Administrator mistake
\item Malicious administrator
\item Physical access
\item System failure
\item System's IP discovered
\item Zero Day
\end{itemize}
\subsection{Components}
The components are the ressources that we are trying to protect inside of our system. We want to prevent an opponent to steal, corrupt or destroy any of the client's assets.
\subsubsection{Hardware components}
\begin{itemize}
\item The data diode
\item The unsecure network
\item The secure network
\end{itemize}
\subsubsection{Software components}
\begin{itemize}
\item The inside administration webpage
\item The outside administration webpage
\item The php and web servers
\item The UDP client and server
\end{itemize}
\subsection{Risk severity}
We define here the severity for each components of the CIA triad.
\subsubsection{Confidentiality}
Severity of loss of confidentiality:
\begin{itemize}
\item More than 1 user credential $->$ Minor
\item More than 100 user credential $->$ Major
\item 1 or more administrator credentials $ ->$ Major

\end{itemize}
\subsubsection{Integrity}
Severity of corrupted packets:
\begin{itemize}
\item More than 0.1\% of the packets are corrupted $->$ Minor
\item More than 1\% of the packets are corrupted $->$ Moderate
\item More than 5\% of the packets are corrupted $->$ Major
\item More than 10\% of the packets are corrupted $->$ Extreme
\end{itemize}
\subsubsection{Availability}
Severity of downtimes:
\begin{itemize}
\item More than 10 minutes $->$ Minor
\item More than 30 minutes $->$ Moderate
\item More than 1 hour $->$ Major
\item More than 6 hour $->$ Extreme
\end{itemize}
\subsection{Threats Assessments}

\subsubsection{Risk 1: Packet sniffing on the diode }
\textbf{Source} \\A malicious person has a physical access to the data diode\\
\textbf{Component} \\The data diode\\
\textbf{Description}\\ This person can use packet sniffing techniques on the unsecure sides of the data diode to obtain the credentials of an administrator. \\
\textbf{Impact}\\
With an administrator access the opponent can modify users and other adminstrators accounts. The confidentiality of the accounts is then corrupted and the availability of our system is also compromised.\\
\begin{tabular}{|c|c|c|}
\hline
Confidentiality & Integrity & Availability \\
\hline
Major & None & Extreme \\
\hline
\end{tabular}\\
\textbf{Likelihood}\\ As we recommand our client to protect the access to the diode, we assume that this risk is unlikely.\\
\textbf{Risk level}\\Medium\\

\subsubsection{Risk 2 : Shut down of the data diode }
\textbf{Source} \\A malicious person has a physical access to the data diode\\
\textbf{Component} \\The data diode\\
\textbf{Description}\\ This person can turn off the data diode. \\
\textbf{Impact}\\
The shut down of the diode cause a downtime of less than 30 minutes.\\
\begin{tabular}{|c|c|c|}
\hline
Confidentiality & Integrity & Availability \\
\hline
None & None & Minor \\
\hline
\end{tabular}\\
\textbf{Likelihood}\\ As we recommand our client to protect the access to the diode, we assume that this risk is unlikely.\\
\textbf{Risk level}\\Low\\

\subsubsection{Risk 3 : Deterioration of the data diode }
\textbf{Source} \\A malicious person has a physical access to the data diode\\
\textbf{Component} \\The data diode\\
\textbf{Description}\\ This person can damage or even destroy the data diode.  \\
\textbf{Impact}\\
Physical damage or complete destruction of the diode will obviously lead to a device replacement creating a several hour downtime.\\
\begin{tabular}{|c|c|c|}
\hline
Confidentiality & Integrity & Availability \\
\hline
None & None & Extreme \\
\hline
\end{tabular}\\
\textbf{Likelihood}\\ As we recommand our client to protect the access to the diode, we assume that this risk is unlikely.\\
\textbf{Risk level}\\Medium\\

\subsubsection{Risk 4 : Flooding of the sender }
\textbf{Source} \\A malicious user send a lot of files in the goal of overloading the sender\\
\textbf{Component} \\The sender server\\
\textbf{Description}\\ The user will overload the system with data pushes. 
\\
\textbf{Impact}\\
 If this person can send more files that the system can handle, our system can become unavailable for several tens of minutes. \\
\begin{tabular}{|c|c|c|}
\hline
Confidentiality & Integrity & Availability \\
\hline
None & None & Moderate \\
\hline
\end{tabular}\\
\textbf{Likelihood}\\ This kind of attack is really popular and easy to make so the event is likely to occur.\\
\textbf{Risk level}\\Medium\\

\subsubsection{Risk 5 : Flooding of the outside webpage}
\textbf{Source} \\A malicious person send a lot of connection requests to the unsecure side login webpage . \\
\textbf{Component} \\The outside webpage\\
\textbf{Description}\\ The outside administration webpage will be overloaded. \\
\textbf{Impact}\\
With an sufficient overload of this system, the administration webpage can be unavailable for several tens of minutes.\\
\begin{tabular}{|c|c|c|}
\hline
Confidentiality & Integrity & Availability \\
\hline
None & None & Moderate \\
\hline
\end{tabular}\\
\textbf{Likelihood}\\ This kind of attack is really popular and easy to make so the probability is likely.\\
\textbf{Risk level}\\Medium\\

\subsubsection{Risk 6 : Brute force on the outside webpage }
\textbf{Source} \\A malicious user tries to obtain administrator rights.\\
\textbf{Component} \\The outside webpage\\
\textbf{Description}\\This malicious user can use brute force technique or dictionnary attack to find an administrator account. \\
\textbf{Impact}\\
This kind of attacks generate a lot of connection attemps. This overload on the webpage can lead to a several minutes downtime. If this attack succeed the user has the ability to modify other user or even administrators accounts. He also can delete all the user accounts so that the system cannot import data anymore.\\
\begin{tabular}{|c|c|c|}
\hline
Confidentiality & Integrity & Availability \\
\hline
Major & None & Major \\
\hline
\end{tabular}\\
\textbf{Likelihood}\\ This kind of attack is really popular and easy to make so the probability is likely.\\
\textbf{Risk level}\\High\\

\subsubsection{Risk 7 : SQL injection on the outside webpage }
\textbf{Source} \\A malicious user tries to connect as an administrator without authorization.\\
\textbf{Component} \\The outside webpage\\
\textbf{Description}\\This person can use SQL injection in the login page in the goal of being connected as an administrator. \\
\textbf{Impact}\\
The user has the ability to modify other user or even administrators accounts. He also can delete all the user accounts so that the system cannot import data anymore.\\
\begin{tabular}{|c|c|c|}
\hline
Confidentiality & Integrity & Availability \\
\hline
Major & None & Major \\
\hline
\end{tabular}\\
\textbf{Likelihood}\\ This kind of attack is really popular and easy to make so the probability is likely.\\
\textbf{Risk level}\\High\\

\subsubsection{Risk 8 : UDP packet loss  }
\textbf{Source} \\UDP is not a secure protocol \\
\textbf{Component} \\The UDP connection between our 2 servers\\
\textbf{Description}\\As UDP is not a secure protocol, some packets can be lost between our two servers. \\
\textbf{Impact}\\
The loss of a UDP packet can lead to uncorrect data.\\
\begin{tabular}{|c|c|c|}
\hline
Confidentiality & Integrity & Availability \\
\hline
None &Moderate & None \\
\hline
\end{tabular}\\
\textbf{Likelihood}\\ With a so small optical fiber, this event is unlikely to occur.\\
\textbf{Risk level}\\Medium\\

\subsubsection{Risk 9 : UDP corrupted packet}
\textbf{Source} \\UDP is not a secure protocol \\
\textbf{Component} \\The UDP connection between our 2 servers\\
\textbf{Description}\\As UDP is not a secure protocol, a UDP packet can be corrupted but we have no way to ask it again. \\
\textbf{Impact}\\
The corruption of a UDP packet can lead to uncorrect data.\\
\begin{tabular}{|c|c|c|}
\hline
Confidentiality & Integrity & Availability \\
\hline
None & Moderate & None \\
\hline
\end{tabular}\\
\textbf{Likelihood}\\ With a so small optical fiber, this event is unlikely to occur.\\
\textbf{Risk level}\\Medium\\

\subsubsection{Risk 10 : Deletion of user accounts }
\textbf{Source} \\A malicious administrator wants to block the import of data.\\
\textbf{Component} \\The outside managment webpage\\
\textbf{Description}\\This administrator can delete all the user accounts \\
\textbf{Impact}\\
Without any authorized user, the system cannot import data anymore. This can create a several hours downtime. \\
\begin{tabular}{|c|c|c|}
\hline
Confidentiality & Integrity & Availability \\
\hline
None & None & Extreme \\
\hline
\end{tabular}\\
\textbf{Likelihood}\\ Having a malicious administrator is unlikely to occur.\\
\textbf{Risk level}\\Medium\\

\subsubsection{Risk 11 : Deletion of administrator accounts }
\textbf{Source} \\A malicious administrator wants to block user management.\\
\textbf{Component} \\The outside managment webpage\\
\textbf{Description}\\This administrator can delete all the other administrator accounts \\
\textbf{Impact}\\
Without any other administrator, we cannot management the users anymore.\\
\begin{tabular}{|c|c|c|}
\hline
Confidentiality & Integrity & Availability \\
\hline
Major & None & None \\
\hline
\end{tabular}\\
\textbf{Likelihood}\\ Having a malicious administrator is unlikely to occur.\\
\textbf{Risk level}\\Medium\\

\subsubsection{Risk 12 : CSRF in the outside webpage}
\textbf{Source} \\A malicious user wants to modify the users or administrators accounts.\\
\textbf{Component} \\The outside managment webpage\\
\textbf{Description}\\This malicious user forges a request so that an administrator session is used to modify accounts.   \\
\textbf{Impact}\\
With this technique, the user can create an administrator account for himself or modify other accounts. He can then block the import of the data to the secure network. \\
\begin{tabular}{|c|c|c|}
\hline
Confidentiality & Integrity & Availability \\
\hline
None & None & Major \\
\hline
\end{tabular}\\
\textbf{Likelihood}\\The opponent needs the unawareness of an administrator to lauch this attack. This event is then unlikely to occur.\\
\textbf{Risk level}\\Medium\\





\subsection{Resume}
\begin{tabular}{|c|c|c|c|c|}
\hline
Treat & Source & Impact& Likelihood & Evaluation \\
\hline
1 &  Packet sniffing & Extreme  & Unlikely & Medium\\
\hline
2 & Shut down of diode & Minor & Unlikely & Low \\
\hline
3 & Deterioration of the diode & Extreme & Unlikely & Medium\\
\hline
4 & Flooding the sender & Moderate & Likely & Medium\\
\hline
5 & Flooding on outside webpage & Moderate & Likely & Medium \\
\hline 
6 &  Brute Force on outside webpage & Major & Likely & High\\
\hline
7 & SQL injection on outside webpage & Major & Likely & High \\
\hline
8 & UDP packet loss & Moderate & Unlikely & Medium \\
\hline
9 & UDP corrupted packet & Moderate & Unlikely & Medium \\
\hline
10 & Deletion of user accounts & Extreme & Unlikely & Medium \\
\hline
11 & Deletion of administrator accounts & Major & Unlikely & Medium \\
\hline
12 & CSRF on the outside webpage & Major & Unlikely & Medium \\
\hline

  
\end{tabular}
\subsection{Countermeasures}
As security has a significative price for a company, we assume that low risks should be accepted.
\subsubsection{Risk 1 : Packet sniffing on the diode}
As the physical access to the diode is out of our scope, we transfer this risk to the client.
\subsubsection{Risk 3 : Deterioration of the diode}
As the physical access to the diode is out of our scope, we transfer this risk to the client.
\subsubsection{Risk 4 : Flooding on the sender}

\subsubsection{Risk 5 : Flooding of the outside webpage}

\subsubsection{Risk 6 : Brute force on outside webpage}
We mitigate this risk by forcing administrators to have a long enough password to make it really hard to brute force.
\subsubsection{Risk 7 : SQL injection on outside webpage}
We avoid this risk by using prepared statement mecanism on the login inputs.
\subsubsection{Risk 8 : UDP packet loss}
We mitigate this risk by pushing 3 times each UDP packets. We then only need one received packet over the three to obtain the data on the secure side.As the chance to have one lost packet is small, the probabilities to have 3 lost packets are nearly null.
\subsubsection{Risk 9 : UDP corrupted packet}
We mitigate this risk with the same techniques than for lost packets but including the UDP checksum. We only need one correct packet over the three to obtain the data on the secure side. The chances to have three corrupted packets are also nearly null.
\subsubsection{Risk 10 : Deletion of user accounts}
As the behaviour and the good will of the employees are under the responsability of the client, we transfer this risk to them.
\subsubsection{Risk 11 : Deletion of administrator accounts}
As the behaviour and the good will of the employees are under the responsability of the client, we transfer this risk to them.
\subsection{Residual risks}






\end{document}

\documentclass[a4paper,11pt]{article}
\usepackage[utf8]{inputenc}
\usepackage{textcomp}
\usepackage{lmodern}
\usepackage{listings}
\usepackage{graphicx}
\usepackage{listings}
\usepackage{color}
\definecolor{lightgray}{rgb}{0.9,0.9,0.9}
\definecolor{darkgray}{rgb}{0.4,0.4,0.4}
\definecolor{purple}{rgb}{0.65, 0.12, 0.82}
\usepackage{url}
\usepackage[top=3cm,bottom=3cm,left=3cm,right=3cm]{geometry}

\title{Royal Military Academy\\
	INFO-Y113 --- Management of Security: \\
	Concept Of Operations v2 \& Risk Analysis}

\author{DANHIER Pierre, LECOCQ Alexis, NYAKI Loïc}

\begin{document}
\maketitle
\newpage
\tableofcontents

\newpage

\section{Introduction}
In recent years, cyber-security has become a primary concern for companies all over the world. No matter the size of the company, data often represent the heart of their business and whether the concern is the secrecy of intellectual property, or users' privacy, the theft of private data bears a huge cost for companies. Be it a monetary cost (lawsuits, fines) or a reputation cost (loss of trust, public outrage). In the case of government agencies, states secrets and other classified information could be stolen by a foreign nation, possibly leading to the loss of lives in conflict zones, loss of political leverage on the international scene, domestic political turmoil and scandals or simply public embarrassment.\\

When trying to protect these sensitive data, a common measure is be to physically isolate the network from the internet, by creating an \textit{air gap}. Acting this way ensures that the data from the network is inaccessible from the outside world. The main issue with this method is that inevitably, some external data or files will at some point need to be imported into the secure network, be it for software update, or simply because some files from the outside are necessary for the people working in the secure network. In that case, a manual import (via USB drive, by connecting an external laptop into the secure network, or by using some other data transfer device) will be necessary.\\

The problem with that method is that it can compromise the security of the secure network. For instance, the data that is manually transferred into the network may have been infected by a malware, or the secure network might already be infected by a virus. In both cases, there is a possibility for some malicious code to exfiltrate data, or to spread a virus outside, by secretly writing on the device that was originally used to import the data into the network, such as USB drives or laptops.\\

As a consequence, we need to build a solution that prevents data leaks while allowing the transfer of files from the outside network into the secure network.

\section{Goals of the project}
\label{sec:goals}
Goals define the general direction of what the organization aims to accomplish, in the long term. Here, we wish to design a system that accomplishes three main goals :

\begin{enumerate}
\item{Create a device that completely prevents the exfiltration of data from a secure network, while allowing data to be transferred from the outside world into that secure network.}
\item{Ensure the availability of the system. The down times should be as minimum as possible.}
\item{Allow specific users to manage and monitor this system through an administration web page.}
\end{enumerate}

In this project, the general solution is imposed and should be a data diode, which will be described in section~\ref{sec:data-diode}.

\section{Scope}
It is important to precisely identify the scope of this project, in relation to our goals and objectives.\\

Our solution is destined to be integrated in an existing system. As such, when considering the security of the system as a whole, we must identify which security aspects fall under our responsibility and which don't. 

\subsection{In scope}
The following elements are in scope, which means that it is our responsibility to make sure that the security of these elements is ensured.

\begin{itemize}
\item{The availability of the service}
\item{The confidentiality of user data and credentials when interacting with the data diode (see section \ref{sec:data-diode})}
\item{The impossibility to exfiltrate data from the secure network.}
\item{The integrity of the data pushed through the data diode}
\end{itemize}

\subsection{Out of scope}
\label{sec:outscope}
The following elements are out of scope. This means that the security of these elements does not fall under our responsibility, but rather under the responsibility of the client, or another third party.

\begin{itemize}
\item{The physical access to the hardware, such as the power button or Ethernet cables}
\item{The physical integrity of the hardware}
\item{The electrical power source}
\item{The security within the secure network, such as the presence of malwares or other viruses}
\end{itemize}


\section{Data Diode}
\label{sec:data-diode}
To ensure the integrity and the confidentiality of the data within our system and the availability of the service, we are going to implement a \textit{data diode}.\\

Just like an electrical diode only conduct electrical current in one direction, a data diode is a networking device that only allows data to flow in one direction. It is composed of two physical servers: one server communicates with the outside network and the other one communicate with the secure network. The two servers are to one another by a single unidirectional fiber optics cable.\\

A fiber optic connection normally uses two cables: one for each direction. In the case of 	 data diode, only one cable is used, allowing the data to flow in one direction. The cable going in the other direction is physically cut. As a consequence, data going through a data diode can only flow in one direction, which is one of the goals required in section \ref{sec:goals}.\\

The one-way communication channel between the two sides of the data-diode forbids the use of usual TCP based protocols (such as HTTP or FTP), as TCP requires bi-directional communication between two parties. As data between the two sides of the data-diode can only flow in one direction, we need data to be send over a protocol that doesn't require bi-directional communication. This can be done by using UDP for the communication between our two servers.\\

However UDP comes with its own limitations, as it doesn't ensure the order at which the paquets arrive, nor does it manage packets loss. These issues will therefore need to be taken into account at the software level.


\begin{figure}
	\includegraphics[scale=0.7]{img/system.png}
	\caption{High level architecture of our data diode.}
\end{figure}


\subsection{Applications}

We propose a JSON API to import files comming from connected devices to a secure network.

\subsubsection{Administration and User Management}
\label{sec:administration}
One of the goals of this project is the creation of a web interface for managing the data-diode. We will implement a web interface on the unsecure side that allows administrators to add new users and a second web interface on the secure side that monitor the arrival of the UDP packets.

\subsubsection{Mockups}
\begin{center}
\begin{tabular}{cc}
\includegraphics[scale=0.5]{img/Login.png} & \includegraphics[scale=0.5]{img/menu.png}\\
\includegraphics[scale=0.45]{img/ftp.png} & \includegraphics[scale=0.5]{img/linkudp.png}\\
\includegraphics[scale=0.5]{img/switch.png} & \\


\end{tabular}
\end{center} 




\subsection{Physical Architecture}
The data-diode is composed of two separate servers connected to each other through a unidirectional fiber optics cable. The server facing the outside network is called \textit{the sender}, the server facing the secure network is called \textit{the receiver}.\\

Each server contains the following components :
\begin{itemize}
\item{Two network interfaces: one for connecting to a network, and one for connecting to the other server}
\item{A fiber optic adapter, to translate the signal coming from the fiber optic cable from light into a signal that can be transmitted through an Ethernet port.}
\end{itemize}

\subsection{Software architecture}
The data-diode is composed of two servers, \textit{the sender} and \textit{the receiver}, which have different roles. 

\subsubsection{The Sender}
The \textit{sender} must be able to receive HTTP requests from the outside network and to transmit data to the \textit{receiver} via UDP.\\

The sender must therefore be comprised of :
\begin{itemize}
\item{A UDP client}
\item{A web server}
\item{A php server}
\end{itemize} 

\subsubsection{The Receiver}
The \textit{receiver} must be able to receive UDP paquets and reconstruct them as a file that will be placed in a database. \\

In summary, the receiver will contain :
\begin{itemize}
\item{A UDP server}
\item{A web server}
\item{A php server}
\end{itemize}

\subsection{Simulating a data-diode}
At first, instead of using a real data-diode, we are going to build the prototype of a data-diode by simulating the actual system through software. The unidirectional nature of the data transfer will be simulated by modifying the IP table in a way that will allow traffic in one direction, and drop all traffic going in the other direction.


\section{Users}
We identify two types of users : the administrators and the simple users.

\subsection{Simple Users}
Simple users are users that push data with their connected devices.

\subsection{Administrators}
The role of the administrators is to manage the system so that it runs smoothly. They have access to all the functionalities of the administration interface as described in section \ref{sec:administration}.

\section{Administration and Management}

\subsubsection{Administration of Simple Users}
Administrators can create and delete user accounts.

\subsubsection{Administration of other Administrators}
Administrators are able to create an administrator account, or to suspend the account of another administrator.

\subsection{data-diode Installation and Configuration}

\subsubsection{Installation}
The data-diode is provided to the client with the programs already installed, and a default administrator account. At first use, the password should be modified.

\subsubsection{Configuration}
The data-diode settings will be specified in a configuration file that can be modified manually.

\section{Risk analysis}
\subsection{Threat sources}
\begin{itemize}
\item Malicious user
\item Administrator mistake
\item Malicious administrator
\item Physical access
\item System failure
\item System's IP discovered
\item Zero Day
\end{itemize}
\subsection{Components}
The components are the ressources that we are trying to protect inside of our system. We want to prevent an opponent to steal, corrupt or destroy any of the client's assets.
\subsubsection{Hardware}
\begin{itemize}
\item The data diode
\item The unsecure network
\item The secure network
\end{itemize}
\subsubsection{Software}
\begin{itemize}
\item The inside administration webpage
\item The outside administration webpage
\item The inside web and php server
\item The outside web and php server
\item The UDP client
\item The UDP server
\end{itemize}
\subsection{Risk severity}
We define here the severity for each components of the CIA triad.
\subsubsection{Confidentiality}
Severity of loss of confidentiality:
\begin{itemize}
\item More than 1 user credential $->$ Minor
\item More than 100 user credential $->$ Major
\item 1 or more administrator credentials $ ->$ Major

\end{itemize}
\subsubsection{Integrity}
Severity of corrupted packets:
\begin{itemize}
\item More than 0.1\% of the packets are corrupted $->$ Minor
\item More than 1\% of the packets are corrupted $->$ Moderate
\item More than 5\% of the packets are corrupted $->$ Major
\item More than 10\% of the packets are corrupted $->$ Extreme
\end{itemize}
\subsubsection{Availability}
Severity of downtimes:
\begin{itemize}
\item More than 10 minutes $->$ Minor
\item More than 30 minutes $->$ Moderate
\item More than 1 hour $->$ Major
\item More than 6 hour $->$ Extreme
\end{itemize}
\subsection{Threats Assessments}
\subsubsection{Risk 1: Outside attack}
\textbf{Source} \\An unauthorized person takes knowledge of the IP adress of the outside server.\\
\textbf{Component} \\The outside web and php server\\
\textbf{Description}\\ This person can send files through the https protocol even if he is not suppose to. \\
\textbf{Impact}\\
\begin{tabular}{|c|c|c|}
\hline 
\emph{Confidentiality} & \emph{Integrity} & \emph{Availability} \\
\hline
Trivial & Major & Trivial\\
\hline
\end{tabular}\\
\textbf{Likelihood} We have no view on the discretion of the client for his IP adresses so we assume that it is likely to occur.\\
\textbf{Risk level} High \\

\subsubsection{Risk 2: packet sniffing from the data diode }
\textbf{Source} \\A malicious person has a physical access to the data diode\\
\textbf{Component} \\The data diode\\
\textbf{Description}\\ This perso can use packet sniffing techniques on the secure and on the unsecure side of the data diode to obtain the credentials of providers , users or administrators \\
\textbf{Impact}\\
\begin{tabular}{|c|c|c|}
\hline
Confidentiality & Integrity & Availability \\
\hline
Major & Major & Trivial \\
\hline
\end{tabular}\\
\textbf{Likelihood}\\ As we recommand our client to protect the access to the diode, we assume that this risk is unlikely.\\
\textbf{Risk level}\\Medium\\

\subsubsection{Risk 3: Shut down of the data diode }
\textbf{Source} \\A malicious person has a physical access to the data diode\\
\textbf{Component} \\The data diode\\
\textbf{Description}\\ This perso can turn off the data diode. \\
\textbf{Impact}\\
\begin{tabular}{|c|c|c|}
\hline
Confidentiality & Integrity & Availability \\
\hline
Trivial & Trivial & Minor \\
\hline
\end{tabular}\\
\textbf{Likelihood}\\ As we recommand our client to protect the access to the diode, we assume that this risk is unlikely.\\
\textbf{Risk level}\\Low\\

\subsubsection{Risk 4: Deterioration of the data diode }
\textbf{Source} \\A malicious person has a physical access to the data diode\\
\textbf{Component} \\The data diode\\
\textbf{Description}\\ This person can damage the data diode.  \\
\textbf{Impact}\\
\begin{tabular}{|c|c|c|}
\hline
Confidentiality & Integrity & Availability \\
\hline
Trivial & Moderate & Extreme \\
\hline
\end{tabular}\\
\textbf{Likelihood}\\ As we recommand our client to protect the access to the diode, we assume that this risk is unlikely.\\
\textbf{Risk level}\\Medium\\

\subsubsection{Risk 5 : Flooding of the receiver }
\textbf{Source} \\A malicious user send a lot of files in the goal of overloading the receiver\\
\textbf{Component} \\The receiver server\\
\textbf{Description}\\ If this person can send more files that the system can handle, our system can become unavailable for a while. \\
\textbf{Impact}\\
\begin{tabular}{|c|c|c|}
\hline
Confidentiality & Integrity & Availability \\
\hline
Trivial & Trivial & Extreme \\
\hline
\end{tabular}\\
\textbf{Likelihood}\\ This kind of attack is really popular and easy to make so the probability is likely.\\
\textbf{Risk level}\\High\\

\subsubsection{Risk 6 : Flooding of the outside webpage}
\textbf{Source} \\A malicious person send a lot of connection request in the goal of overload this application. \\
\textbf{Component} \\The outside webpage\\
\textbf{Description}\\ If this person can send more requests that the application can handle, the administration webpage can become unavailable \\
\textbf{Impact}\\
\begin{tabular}{|c|c|c|}
\hline
Confidentiality & Integrity & Availability \\
\hline
Trivial & Trivial & Moderate \\
\hline
\end{tabular}\\
\textbf{Likelihood}\\ This kind of attack is really popular and easy to make so the probability is likely.\\
\textbf{Risk level}\\Medium\\

\subsubsection{Risk 7 : Brute force on the outside webpage }
\textbf{Source} \\A malicious user tries to find an administrator identification.\\
\textbf{Component} \\The outside webpage\\
\textbf{Description}\\This person can use brute force technique or dictionnary attack to find an administrator account. \\
\textbf{Impact}\\
\begin{tabular}{|c|c|c|}
\hline
Confidentiality & Integrity & Availability \\
\hline
Medium & Trivial & Medium \\
\hline
\end{tabular}\\
\textbf{Likelihood}\\ This kind of attack is really popular and easy to make so the probability is likely.\\
\textbf{Risk level}\\High\\

\subsubsection{Risk 8 : SQL injection on the outside webpage }
\textbf{Source} \\A malicious user tries to connect as an administrator.\\
\textbf{Component} \\The outside webpage\\
\textbf{Description}\\This person can use SQL injection in the login page in the goal of being connected as an administrator. \\
\textbf{Impact}\\
\begin{tabular}{|c|c|c|}
\hline
Confidentiality & Integrity & Availability \\
\hline
Medium & Trivial & Trivial \\
\hline
\end{tabular}\\
\textbf{Likelihood}\\ This kind of attack is really popular and easy to make so the probability is likely.\\
\textbf{Risk level}\\Medium\\

\subsubsection{Risk 9 : UDP packet loss  }
\textbf{Source} \\The UDP connection between our two servers \\
\textbf{Component} \\The data diode\\
\textbf{Description}\\As UDP is not a secure protocol, some packets can be lost between our two servers. \\
\textbf{Impact}\\
\begin{tabular}{|c|c|c|}
\hline
Confidentiality & Integrity & Availability \\
\hline
Trivial & Major & Trivial \\
\hline
\end{tabular}\\
\textbf{Likelihood}\\ With a so small optical fiber, the chance to loose a UDP packet are rare.\\
\textbf{Risk level}\\Medium\\

\subsubsection{Risk 10 : Deletion of user account }
\textbf{Source} \\A malicious administrator \\
\textbf{Component} \\The outside managment webpage\\
\textbf{Description}\\This administrator can delete all the user's accounts \\
\textbf{Impact}\\
\begin{tabular}{|c|c|c|}
\hline
Confidentiality & Integrity & Availability \\
\hline
Trivial & Extreme & Extreme \\
\hline
\end{tabular}\\
\textbf{Likelihood}\\ Unlikely\\
\textbf{Risk level}\\Medium\\




\subsection{Resume}
\begin{tabular}{|c|c|c|c|c|}
\hline
Treat & Source & Impact& Likelihood & Evaluation \\
\hline
1 & Known IP adress & Major & Likely & High\\
\hline
2 &  Direct packet sniffing & Major & Unlikely & Medium\\
\hline
3 & Shut down of diode & Minor & Unlikely & Low \\
\hline
4 & Damage to diode & Extreme & Unlikely & Medium\\
\hline
5 & DoS attack on receiver & Extreme & Likely & High\\
\hline
6 & Flooding on outside webpage & Moderate & Likely & Medium \\
\hline 
7 &  Brute Force on outside webpage & Moderate & Likely & Medium\\
\hline
8 & SQL injection on outside webpage & Moderate & Likely & Medium \\
\hline
9& UDP packet loss & Major & Rare & Medium \\
\hline
10 & Deletion of user account & Extreme & Unlikely & Medium \\
\hline
  
\end{tabular}
\subsection{Countermeasures}
As security has a significative price for a company, we assume that low risks should be accepted.
\subsubsection{Risk 1 : Outside attack}
\subsubsection{Risk 2 : Packet sniffing on the diode}
We mitigate the risk by encrypting the credentials while they're on the network.
\subsubsection{Risk 4 : Damage to the diode}
As the physical access to the diode is out of our scope, we transfer this risk to the client.
\subsubsection{Risk 5 : DoS of the receiver}

\subsubsection{Risk 6 : Flooding of the outside webpage}
\subsubsection{Risk 7 : Brute force on outside webpage}
\subsubsection{Risk 8 : SQL injection on outside webpage}
\subsubsection{Risk 9 : UDP packet loss}
\subsubsection{Risk 10 : Deletion of user accounts}






\end{document}
